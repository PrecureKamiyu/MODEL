\section{问题重述}
在制造集成电路板时,要在回焊炉中加热印刷电路板,以使电子元件被高温焊接至电路板上。整个焊接过程中,通过恰当地设置回焊炉各温区温度,可使电路板的焊接工艺达到较高水准。
基于以上背景信息和及题目和附件给出的数据信息,需建立数学模型解决以下问题:
\subsection{问题1}
已知传送带的过炉速度为 \(78\,\mathrm{cm}/\mathrm{min}\),且设定小温区\(1{\sim} 5\)的温度为\(173 ^{\circ} \mathrm{C}\),小温区6的温度为\(198^{\circ}\mathrm{C}\),小温区7的温度为\(230^{\circ}\mathrm{C}\),小温曲\(8{ \sim }9\)的温度为\(257^{\circ}\mathrm{C}\),要求建立数学模型求解焊接区域温度变化规律,作出这种情况下的炉温曲线,给出小温区3、6、7中点、小温区8末端且位于焊接区域中心位置处的温度,并保存每隔\(0.5\, \mathrm{s}\)的焊接区域中心温度至\texttt{result.csv}中。
\subsection{问题2}
\subsection{问题3}
\subsection{问题4}
