\documentclass[../main.tex]{subfiles}

\begin{document}

\section{问题重述}
\subsection{问题背景}
回流焊接是指利用焊膏将一或多个电子元件连接到接触垫上之后,透过控制加温来熔化焊料以达到永久接合,用回焊炉等不同加温方式来进行焊接。

回流焊接的程序目的在于逐步熔化焊料与缓慢加热连接界面,避免急速加热而导致电子元件的损坏。通常分为四个阶段,称为 “区(Zone)”,每一个区都拥有各自的温度曲线:“预热”、“浸热”、“回流” 与 “冷却”。

在回流焊机的使用中,回流焊温度曲线的设定至关重要且难以掌握。实践表明,严格控制温度过程可大大减少焊接缺陷,提高产品的直通率。不良的温度曲线可以导致各种回流焊接缺陷。在实际生产之中,常用测量仪器随电路板进入回焊炉中进行测量,并且不同焊膏有不同性质,最佳炉温曲线不同,因此如何构建出回流焊接过程之中炉温曲线的数学模型来提升焊接质量,已成为一个重要议题。

\subsection{问题提出} % ?
基于以上背景信息和及题目和附件给出的数据信息,需建立数学模型解决以下问题:

\paragraph{问题1}
已知传送带的过炉速度为 \(78\,\mathrm{cm}/\mathrm{min}\),且设定小温区 \(1{\sim}5\) 的温度为\(173^{\circ} \mathrm{C}\),小温区6的温度为\(198^{\circ}\mathrm{C}\),小温区7的温度为\(230^{\circ}\mathrm{C}\),小温曲\(8{ \sim }9\)的温度为\(257^{\circ}\mathrm{C}\),要求建立数学模型求解焊接区域温度变化规律,作出此情况下的炉温曲线,给出小温区3、6、7中点、小温区8末端且位于焊接区域中心位置处的温度,并保存每隔\(0.5\, \mathrm{s}\)的焊接区域中心温度至\texttt{result.csv}中。

\paragraph{问题2}
设定小温区\(1{\sim}5\)的温度为 \(182^{\circ}\mathrm{C}\),小温区6的温度为\(203^{\circ}\mathrm{C}\),小温区7的温度为\(237^{\circ}\mathrm{C}\),小温区\(8{\sim}9\)的温度为\(254^{\circ}\mathrm{C}\),要求给出传送带过炉速度的最大值。

\paragraph{问题3}
要求焊接区域中心温度超过\(217^{\circ}\mathrm{C}\)的时间尽可能短,温度的最大值尽可能小,炉温曲线上温度从大于\(217^{\circ}\mathrm{C}\)至最大值部分所覆盖的面积取最小值,在上述约束下确定最优炉温曲线,并给出各温区设定的温度和传送带过炉速度,以及在上述参数下大于 \(217 ^{\circ}\mathrm{C}\) 至最大值温度部分所覆盖的面积的最小值。

\paragraph{问题4}
在问题三的基础上,对于超过\(217 ^{\circ}\mathrm{C}\)的炉温曲线部分,额外要求其以峰值温度为中心线的左右两侧炉温曲线对称,要求给出最优炉温曲线及其各项指标。
\end{document}
