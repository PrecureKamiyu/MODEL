\documentclass[../main.tex]{subfiles}

\begin{document}

\section{问题分析}

\paragraph{问题1}
问题1可分解为求回焊炉内的温度分布和已知回焊炉内温度分布求炉温曲线两部分,由于涉及空气之间、空气与焊锡之间的传热过程,因此考虑建立热传导模型。考虑到题目中缺少小温区内焊锡的热传导系数 \(A_i\)、焊锡与空气间的热对流系数\(h_i\),因而需要先利用热传导方程和附件中给定条件下的炉温曲线数据拟合出各个温区内部的参数\(A_{i}\), \(h_{i}\),再代入问题1给定条件下的热传导方程中,最终得到欲求的炉温曲线和炉内各点温度。具体步骤如下:
\begin{enumerate}
\item 列出回焊炉内空气的热传导方程,确定边界条件,由于考虑稳定状态,无需确定初始条件。
\item 在通过1求得回焊炉内温度分布的基础上,列出电路板上的焊锡所满足的热传导方程,确定初始条件和边界条件,此时方程含有未知的焊锡的热传导系数\(A_{i}\)、焊锡与空气间的热对流系数\(h_{i}\)。
\item 对\(A_{i}, h_{i}\)分别赋值,代入2中求得对应参数下的炉温曲线,并与附件给出的每个小温区内对应的数据进行比较,找出对每个小温区对应的炉温曲线具有最优拟合效果的参数 \(\hat A _{i} , \hat h_{i}\)。
\item 类似1、2,列出问题1给定条件下回焊炉内空气的热传导方程以及电路板上的焊锡所满足的热传导方程,确定边界条件和初始条件,并将3中找出的各个温区内部的最优参数 \(\hat A _{i} , \hat h _{i}\)代入方程中,求解即可得到欲求的炉温曲线和炉内各点温度。
\end{enumerate}

\paragraph{问题2}
依照题目给出的对于炉温曲线的数个限制条件,使用二分法搜索出各个限制条件下最大的过炉速度,随后在其中选出能够满足所有限制条件的最大的过炉速度。

\paragraph{问题3}
将超过 \(217 ^{\circ}\mathrm{C}\) 到峰值温度所覆盖的面积最小作为优化条件,在满足制程界限这一约束条件的范围内进行二分搜索,以选出满足约束条件且阴影部分面积取最小值的最优炉温曲线,并给出各温区对应的设定温度、传送带对应的过炉速度以及阴影部分对应的最小面积。

\paragraph{问题4}
问题四有两个优化条件:以峰值温度为中心线的两侧超过\(217 ^{\circ}\mathrm{C}\)的炉温曲线尽可能对称、超过\(217 ^{\circ}\mathrm{C}\)到峰值温度所覆盖的面积尽可能小,同时依然要满足制程界限这一约束条件。针对炉温曲线的对称性,应先确定其度量指标。

由于存在两个优化条件,因此要使用熵权值法确定这两个优化条件各自的权重,再在满足制程界限这一约束条件的范围内进行二分搜索,以选出使得加权后的约束条件之和取最小值得最优炉温曲线,并给出各温区对应的设定温度、传送带对应的过炉速度、度量指标对应的值。
\end{document}
