\section{符号说明}
\begin{table}[H]
\centering
\begin{tabular}{lcc}\hline
符号				&释义					&单位\\ \hline
\\[-1em]
\(d\)				&小温区的长度 & cm\\
\(\varDelta d\)&温区之间的间隙的长度 & cm \\
\(d_{0}\)		&炉前后区域的长度 & cm\\
\(h\)				&焊接区域厚度 & mm\\
\(v\)				&电路板温度 & \({}^{\circ}\mathrm{C}\)\\
\(u\)				&空气温度 &\({}^{\circ}\mathrm{C}\)\\
\(2l\)			&上下温区表面的之间的距离 & cm\\
\(c\)				&电路板的比热容 & \(\mathrm{J}/(\mathrm{kg}\cdot \mathrm{K})\)\\
\(\rho\)		&电路板的密度 & \(\mathrm{kg}/\mathrm{m}^{3}\)\\
\(V\)				&过炉速度 & \(\mathrm{cm}/\mathrm{min}\)\\
\(T_{0}\)		&车区温度 & \({}^{\circ}\mathrm{C}\)\\
\(T_{1\sim 5}\)&温区 \(1 \sim 5\) 的温度 & \({}^{\circ}\mathrm{C}\)\\
\(T_{6}\)		&温区 \(6\) 的温度 & \({}^{\circ}\mathrm{C}\)\\
\(T_{7}\)		&温区 \(7\) 的温度 & \({}^{\circ}\mathrm{C}\)\\
\(T_{8\sim 9}\)&温区 \(8 \sim 9\) 的温度 & \({}^{\circ}\mathrm{C}\)\\
\(T_{10\sim 11}\)&温区 \(10 \sim 11\) 的温度 & \({}^{\circ}\mathrm{C}\)\\[-1em]
\\ \hline
\end{tabular}
\end{table}
