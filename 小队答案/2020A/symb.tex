\documentclass[../main.tex]{subfiles}

\begin{document}
\section{符号说明}
\begin{table}[H]
\centering
\begin{tabular}{lcc}\hline
\makebox[5em][l]{符号}				&\makebox[10em]{释义}					&\makebox[5em][c]{单位}\\ \hline \\[-15pt]\hline
\\[-1em]
\(d\)				&小温区的长度 & cm\\
\\[-1em]
\(\varDelta d\)&温区之间的间隙的长度 & cm \\
\\[-1em]
\(d_{0}\)		&炉前后区域的长度 & cm\\
\\[-1em]
\(\it height\)&焊接区域厚度 & mm\\
\\[-1em]
\(v\)				&电路板温度 & \({}^{\circ}\mathrm{C}\)\\
\\[-1em]
\(u\)				&空气温度 &\({}^{\circ}\mathrm{C}\)\\
\\[-1em]
\(A\)			&焊锡的热扩散系数 & \(\mathrm{m}^{2}/ \mathrm{s}\)\\
\\[-1em]
\(h\)			&焊锡和空气之间的热对流系数 &  \(\mathrm{W}/ (\mathrm{m}^{2}\cdot \mathrm{K})\)\\
\\[-1em]
\(k\)			&焊锡的热导率	& \(\mathrm{W} / (\mathrm{m} \cdot \mathrm{K})\) \\
\\[-1em]
\(V\)				&过炉速度 & \(\mathrm{cm}/\mathrm{min}\)\\
\\[-1em]
\(T_{0}\)		&车区温度 & \({}^{\circ}\mathrm{C}\)\\
\\[-1em]
\(T_{1\sim 5}\)&温区 \(1 {\sim} 5\) 的温度 & \({}^{\circ}\mathrm{C}\)\\
\\[-1em]
\(T_{6}\)		&温区 \(6\) 的温度 & \({}^{\circ}\mathrm{C}\)\\
\\[-1em]
\(T_{7}\)		&温区 \(7\) 的温度 & \({}^{\circ}\mathrm{C}\)\\
\\[-1em]
\(T_{8\sim 9}\)&温区 \(8 {\sim} 9\) 的温度 & \({}^{\circ}\mathrm{C}\)\\
\\[-1em]
\(T_{10\sim 11}\)&温区 \(10 {\sim} 11\) 的温度 & \({}^{\circ}\mathrm{C}\)\\[-1em]
\\ \hline
\end{tabular}
\end{table}
\end{document}
