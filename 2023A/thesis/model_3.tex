\subsection{问题3的模型建立和求解}
\subsubsection{模型建立}
对于定日镜场的布局,本文采用与问题二相同的布局策略。因此,为确定定日镜场的布局,需要将\(y_{R}\),\(a\),\(\Delta r\)作为决策变量。根据问题二中的分析,参数取值同样应满足\(x_{R} = 0\),\(y_{R} \in [-350, 0]\),且相邻定日镜圈层之间的半径差设定为\(\Delta r\),同一圈上相邻定日镜中心的距离设定为\(a+5\) m,最内圈层的点距离中心塔的距离大于100m

由于创建定日镜场的函数{\it getNumHelio}接收的\(a\)参数唯一,因此不同定日镜的镜面宽度必须相同,即所有定日镜的\(a\)参数均为同一个决策变量。

由于定日镜数量庞大,难以将每一个定日镜的安装高度、镜面高度均作为决策变量进行优化,因此本文采用线性分布模型,即仅将最内层的定日镜安装高度、镜面高度作为决策变量,其余位置定日镜的安装高度、镜面高度、沿圆的径向作线性分布。若记最内层定日镜的安装高度为\(d\)、镜面高度为\(b\),根据线性分布模型,相邻两层之间的安装高度差为\(\Delta d\)、镜面高度差为\(\Delta b\),则定日镜的安装高度、镜面高度随层数\(k\)的分布如下:
\begin{equation}
d_{k} = d + \Delta d\cdot (k-1)\\
b_{k} = b + \Delta b\cdot (k-1)
\end{equation}

针对吸收塔北侧的多数定日镜进行分析:本文通过调整\(\Delta d > 0\)来确保外圈定日镜的反射光线不被内圈的挡住,
以及内圈定日镜不会挡住外圈定日镜的入射光线,使得阴影遮挡损失最小。在此基础上,就可以相应地增加每一圈定日镜的面积。对于每一圈定日镜面积所增加的幅度,本文根据当地一年中正午太阳高度角 \(\alpha_{s}\)的变化情况,再结合定日镜场的排布,得到如下分析图
\begin{figure}[H]
\centering
\input 30_70.tex
\caption{太阳高度角最大最小时塔顶投射的距离示意图}\label{30_70}
\end{figure}

由上图可知,对于位于塔顶投射点的一圈定日镜,余弦效率达到极大值,由于大部分时间内,该定日镜圈都在吸收塔100m内,于是假设内圈的定日镜的余弦效率比外圈的大,因此本文通过调整\(\Delta b \le 0\) 来确保内圈定日镜的面积更大,使得对于同一个的定日镜场布局,定日镜场的
\(E_{\mathrm{field}} / S\)
最大。


设定日镜场的总层数为\(N_{\text{层数}}\),根据题目中给出的定日镜尺寸的限制条件,应满足
\[
\begin{cases}
\min d_{k} = d \ge 2\\
\max d_{k} = d + (N_{\text{层数}}-1) \cdot \Delta d \le 6\\
\min b_{k} = b + (N_{\text{层数}}-1) \cdot \Delta b \ge 2\\
\max b_{k} = b \le a \le 8
\end{cases}
\]

根据题目信息,定日镜镜场的额定年平均输出热功率 \(P_{\mathrm{out}} \) 为 60MW,而优化目标为单位镜面面积年平均输出热功率,问题可以转化为在额定功率下定日镜镜场的总镜面面积\(S\)最小。

综上,问题三的优化目标函数为
\begin{equation}
\min S = ab N,
\end{equation}
决策变量为
\begin{equation}
\{ y_{R} , a, b , d, \Delta r, \Delta b , \Delta d\},
\end{equation}
约束条件为
\begin{equation}
s.t.
\begin{cases}
N = {\it getNumHelio} (y_{R} , a, \Delta r)\\
P_{\mathrm{out}} = 60 \mathrm{MW}\\
2 \le d \le 6 - \Delta d \cdot (N_{\text{层数}}-1)\\
2 - \Delta d \cdot (N_{\text{层数}}-1) \le b \le a \le 8\\
\displaystyle\frac{b_{k}}{2} \sin (\max \varphi_{\text{水平转角}k})
\le d_{k},\quad k \in [1, N]
\end{cases}
\end{equation}

\subsubsection{模型求解}
使用遗传算法求得最优参数如下:
\[
\begin{cases}
a        = 6.365\\
y_{R}    = -197.03\\
\Delta r = 12.96\\
\Delta b = -0.002\\
\Delta d = 0.0012\\
b_{0}    = 4.386\\
d_{0}    = 4.314
\end{cases}
\]
\begin{table}[H]
\caption{\kaishu 问题3每月21日平均光学效率及输出功率}
\input tab_3_1.tex
\end{table}
%
\begin{table}[H]
\caption{\kaishu 问题3年均光学效率及输出功率表}
\input tab_3_2.tex
\end{table}
%
\begin{table}[H]
\caption{\kaishu 问题{\rm 3}设计参数表}
\input tab_3_3.tex
\label{tab_3_3}
\end{table}
