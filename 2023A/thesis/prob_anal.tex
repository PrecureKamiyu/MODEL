\section{问题分析}
为了计算定日镜的光学效率,需要确定阴影遮挡效率,余弦效率,大气透射率,集热器截断效率,以及镜面反射率。

其中余弦效率是由镜面法向和太阳入射光未必平行所造成的影响造成的,大气透射损失是太阳光在大气中传播时大气中的粉尘、颗粒所造成的影响。阴影损失有三种因素造成 1) 中心塔投射到平面镜上造成的阴影;2) 入射太阳光被相邻定日镜所遮挡造成的阴影;3) 定日镜反射的太阳光线被相邻定日镜所遮挡导致的挡光。截断损失是由反射光线并未射到集热器上导致的损失。

为了得到阴影遮挡效率,需要知道在给定了时间的情况下,对于定日镜上的任意一个点是否会被阴影遮挡。需要按顺序确定上面三种因素是否会造成遮挡:根据时间确定太阳光入射角度、反射光角度,检测入射光线是否和中心塔、相邻定日镜有交点,检测反射光线是否和相邻定日镜有交点。

为了测量截断效率,需要考虑入射反射光线都是光锥,计算定日镜上某点反射的光斑射到集热器上的面积比值。
\subsection{问题1}
根据题意,所有定日镜的位置、安装高度、镜面尺寸已知,要求算出此设置之下镜场的年平均输出热功率、镜场的年平均光学效率,为此,本文拟采用蒙特卡洛方法,采用光线追踪,模拟光线是否发生阴影遮挡损失和集热器截断损失,以此得出定日镜的阴影遮挡效率和截断效率。
由光学效率计算公式,得到定日镜的在某时刻光学效率。
模拟当地时间每个月的21日的9:00、10:30、12:00、13:30、15:00每个定日镜的光学效率,以此得到所有定日镜的年平均光学效率。
进而得出整个镜场的年平均光学效率。最后根据法向直接辐射辐照度DNI计算公式、输出热功率公式得出镜场的年平均输出热功率。
\subsection{问题2}
问题2是一个优化问题。根据题意,定日镜的位置并没有给出,需要确定定日镜场的布局。本文拟采用均匀布局,由中心塔作为圆心,在一定半径的圆周上等距地排列定日镜,并且相邻定日镜圈上之间的半径之差相等。
