\documentclass{myclass}
\title{title}
\tihao{A}
\baominghao{2101105}
\schoolname{hit}
\membera{A}
\memberb{B}
\memberc{C}
\supervisor{my father}
\yearinput{2023}
\monthinput{13}
\dayinput{today}
\usepackage{float}
\usepackage{listings}
\usepackage{tikz}
\usepackage{subfiles}
\pagestyle{plain}
\begin{document}
\begin{abstract}
\textbf{针对问题1},本文建立了基于蒙特卡洛模拟的\textbf{几何关系}求解模型。首先根据各日期和当地时间确定太阳高度角和方位角,以此确定入射光线的方向向量,根据所有的定日镜的坐标,调整镜面的朝向使得反射光线落在吸收塔集热器中心。
遍历所有的定日镜,在每一个定日镜上运用\textbf{蒙特卡洛原理}模拟生成大量随机点,运用\textbf{旋转矩阵}实现坐标变换,分为三种情况判断该点是否造成阴影遮挡损失:入射光线是否被塔挡住,入射光线是否被其他定日镜挡住,反射光线是否被其他定日镜挡住。
若该点未造成阴影遮挡损失,则进一步用\textbf{蒙特卡洛}光线追迹法求解该点的截断效率,在以反射点为顶点、主反射光线为轴的光锥内生成大量随机光线,判断其是否落在吸收塔集热器上。通过上述步骤求得定日镜的阴影遮挡效率和截断效率,在计算大气透射率和余弦效率之后,可得出各个定日镜的光学效率。最后求得定日镜场的年平均光学效率为 \textbf{64.93\%},年平均热功率为 \textbf{39.71MW},单位面积镜面年平均输出热功率为 \textbf{0.6322\kern 1ptkW\kern -1pt/m}\({}^{\mathbf{2}}\)。

\textbf{针对问题2} 本文建立了如下单目标\textbf{优化模型}:
首先,定日镜镜场的排布为\textbf{均匀布局},即以吸收塔作为圆心,在一定半径的圆周上等距地排列定日镜,并且相邻定日镜圈上之间的半径之差相等。
其次将题目所需要的单位镜面面积年均输出热功率最大,转化为以镜场总镜面面积最小为优化目标,吸收塔位置坐标、镜面尺寸、安装高度、相邻定日镜圈半径增量为决策变量,以题中镜面尺寸、安装高度以及相邻定日镜底座中心距离要求,额定年平均输出热功率60MW作为约束条件,采用\textbf{遗传算法}进行最优解的搜索,最终得到镜场镜面总面积最小值为 \textbf{77758}\(\mathbf{m}^{\mathbf{2}}\),单位面积年平均输出热功率最大值为
\textbf{0.767}\kern -1pt\(\mathbf{kW\kern -1pt/m^{2}}\)。

\textbf{针对问题3} 本文基于问题2的排布方式,对优化模型进行改进。
对于各个定日镜的安装高度和镜面尺寸,使用线性增长模型进行修正,即各个定日镜的安装高度和镜面尺寸与其到吸收塔的距离成正比。
% TODO revise require
最终得到
% TODO ans require
\end{abstract}
\subfile{prob_retell.tex}
\subfile{prob_anal.tex}
\subfile{symb.tex}
\subfile{assump.tex}
\subfile{model.tex}
\subfile{model_2.tex}
\subfile{model_3.tex}
\subfile{comment.tex}
\subfile{appendix.tex}
\end{document}
