\documentclass[../main.tex]{subfiles}
\begin{document}
\section{模型建立和求解}
\subsection{问题1}
\subsubsection{模型准备}
\paragraph{太阳高度角和太阳方位角}
记 \(\alpha _{s}\) 为太阳高度角,\(\gamma _{s}\)为太阳方位角。已知:
\begin{equation}
\sin \alpha _{s} = \cos \delta \cos \varphi \cos \omega + \sin \delta \sin \varphi
\end{equation}
\begin{equation}
\cos \gamma _{s} = \frac{\sin \delta -\sin \alpha_{s} \sin \varphi}{\cos \alpha _{s} \cos \varphi}
\end{equation}
记 \( \varphi\) 为纬度,\(\omega\) 为太阳时角,\(ST\) 为当地时间, \(\delta\) 为太阳赤纬角,有
\begin{equation}
\omega = \frac{\pi}{12} ( ST - 12)
\end{equation}
\begin{equation}
\sin \delta = \sin \frac{2 \pi D}{365} \sin \big(\frac{2\pi}{360} \cdot 23.45\big)
\end{equation}
设太阳中心发出的光线的方向向量为 \(\vec\lambda _{i} = (x,  y , z)\)。若将\(\vec \lambda_{i}\)单位化,容易知道
\begin{equation}
\begin{cases}
x = - \cos \alpha_{s} \sin \gamma _{s} \\
y = - \cos \alpha_{s} \cos \gamma_{s} \\
z = - \sin \alpha_{s}
\end{cases}
\end{equation}
\paragraph{相关公式}
法向直接辐射照度 DNI 按照下面公式近似计算:
\[
\begin{aligned}
\mathrm{DNI} = G_{0} \bigg[ a + b \exp{- \frac{c}{\sin \alpha_{s}}}\bigg]\\
a = 0.4237 = 0.00821 (6 - H) ^{2} \\
b = 0.5055 + 0.00595(6.5 - H) ^{2} \\
c = 0.2711 + 0.01858 (2.5 - H) ^{2}
\end{aligned}
\]
镜场的输出热功率为 \(E_{\mathrm{field}}\) 为
\[
E_{\mathrm{field}} = \mathrm{DNI} \cdot \sum _{i} ^{N} A_{i} \eta _{i}
\] 
定日镜光学效率\(\eta\)为
\begin{equation}
\eta = \eta _{\mathrm{s b}} \eta _{\cos} \eta _{\mathrm{at}} \eta _{\mathrm{trunc}} \eta _{\mathrm{ref}}
\end{equation}
其中 \(\eta _{\mathrm{s b}}\) 为阴影遮挡效率。阴影遮挡损失是指由于阴影或者是遮挡导致的效率损失,\(\eta _{\cos}\) 是余弦效率,大气透射率 \(\eta _{\mathrm{at}}\) 由下面公式确定:
\begin{equation}
\eta _{\mathrm{at}} = 0.99321 - 0.0001176 d _{\mathrm{HR}} + 1.97 \times 10 ^{-8} \times d _{\mathrm{HR}} ^{2}
\end{equation}
其中 \(d _{\mathrm{HR}}\)是镜面中心到集热器中心的距离。截断效率为\(\eta _{\mathrm{trunc}}\) 。因为部分反射光束没有照射到中心塔塔顶的集热器导致效率损失。

设入射光线为 \(\vec S\),反射光线为 \(\vec R\),余弦效率使用下面公式计算:
\begin{equation}
\eta _{\cos} = \cos \big(\arccos (\vec R \cdot \vec S) / 2\big)
\end{equation}
\paragraph{法向量}
% TODO
\paragraph{阴影遮挡效率的计算}
阴影遮挡效率 \(\eta _{\mathrm{s b}}\) 在进行定日镜光学效率的计算之中计算比较复杂的部分。为了计算阴影遮挡效率,需要计算定日镜上某点是否有阴影遮挡损失,而阴影遮挡损失分为以下三种情况:
% TODO cite require
\begin{itemize}
\item 入射光线被塔挡住
\item 入射光线被其他定日镜挡住(阴影)
\item 反射光线被其他定日镜挡住(挡光)
\end{itemize}
对于定日镜 A,建立相对于定日镜A的坐标系,以镜面中心为原点水平方向为 \(x\) 轴,平行于镜面、垂直于 \(x\) 轴方向为 \(y\) 轴,以垂直于镜面向上为 \(z\) 轴。
已知定日镜 A 上的一点 \(N_{A} = (x_{A}, y_{A})\),和入射光线 \(\vec {\lambda _{i}}\)。
判断入射光线是否被中心塔遮挡,需将定日镜 A 坐标系之中的点 \(N_{A}\) 转化为相对于地面的坐标 \(N\),随后根据 \(\vec {\lambda _{i}}\),计算入射光线的反向延长线与中心塔塔顶所在水平面的交点 \(N_{h} = (x _{h} , y_{h}, h)\),其中 \(h\) 是中心塔塔顶所在水平面高度。
若要得出是否被遮挡,可以直接联立直线方程和中心塔圆柱方程,查看是否有解。

由于直接联立方程计算量过大,本文根据得出的 \(N_{h}\) 点坐标,通过几何关系判断入射是否被遮挡。
\begin{itemize}
\item 若是入射光线水平方向的夹角过大,则不会被遮挡;
\item 若是 \(N_{h}\) 更加靠近定日镜一侧,则不会被遮挡;
\item 否则入射光线会被遮挡。
\end{itemize}
下面给出详细解释。过 \(N\) 作平面和中心塔圆柱相切,设相切作成的直线和中心塔塔顶所在平面的交点为 \(R\)。设中心塔在塔顶所在平面的圆的圆心为 \(O\),\(A\) 为\(N\)在中心塔塔顶所在平面投影,\(\vec \lambda _{i} '\)为 \(\vec \lambda _{i}\) 在平面内的投影。如下图所示
\input fig_ta.tex
若有
\begin{equation}
\cos \langle \overrightarrow {OA}, \vec {\lambda _{i} '} \rangle \le \frac{\vert AR \vert }{\vert OA \vert }
\end{equation}
则入射光线和在水平方向上和 \(OA\)的夹角过大,无论入射光线的高度角如何,其都不会和中心塔有相交。

否则需要根据 \(N_{h}\) 进行判断,若有
\begin{equation}
\vert {\it A N}_{h}\vert \le \vert {\it AR\,} \vert
\end{equation}
且 \( N_{h}\) 不在中心塔圆内,则说明入射光线不会被中心塔遮挡。否则被中心塔遮挡。
\paragraph{光锥}
因为太阳光并不是平行光,在定日镜上的每个点发出的反射光实际是一个光锥,其半角展宽为 \(4.65 \, \mathrm{m}\mathrm{rad}\),在太阳发散角内均匀追迹若干光线作为入射光。因为定日镜是平面镜,于是入射光和反射光完全对称。

以太阳中心发出的光线的反射光 \(\vec\lambda _{o} \) 为轴心,建立关于 \(\vec \lambda _{o}\) 的光锥坐标系
% TODO

对于光锥内的一束光线 \(\vec S _{0}\),记 \(\tau\) 为光线和 \(\vec\lambda _{o}\) 所成的角,其取值范围为 \([0, 4.65]\),\(\sigma\) 为光线和 \(X\) 轴所成的角,其取值范围为 \([0 , 2\pi ]\),于是光线的坐标可以表示为:
\[
(\sin \sigma \cos \tau, \sin \sigma \sin \tau, \cos \sigma)
\]
通过变换矩阵,可以将光线在光锥坐标系下的坐标转换为地面坐标系下的坐标 \(\vec S\)。
% TODO
\paragraph{截断效率的计算}
计算截断效率,需要确定光锥之中的某光线是否有截断损失,若是其没有打在集热器上,则该光线发生了截断损失。为了判断截断损失,采取的思路和上文确定中心塔造成的阴影损失的思路类似。

已知反射点 \(N\) 和 反射光线 \(S\),确定光线 1) 和中心塔塔顶所在平面的交点 \(N'\);2) 和集热器底部所在的平面的交点 \(N''\)。剩余符号沿用上文,如下图所示:

分三个步骤进行截断损失的判断:
\begin{enumerate}
\item 若是 \( \displaystyle\cos \langle \overrightarrow{OA}, \vec S' \rangle\le \frac{\vert AR \vert}{\vert OA \vert}\),则发生截断损失
\item 若是 \(\vert AN'\vert \le \vert AR \vert\)且 \(N'\) 不在圆内,则发生截断损失
\item 若是 \(N''\) 在圆 \(O\) 内,或是 \(\vert AN'' \vert \ge \vert AR \vert\)
\end{enumerate}


\begin{figure}[H]
\centering
\begin{subfigure}[b]{0.4\textwidth}
\centering
\input fig_trunc.tex
\caption{nihao}
\end{subfigure}
\begin{subfigure}[b]{0.4\textwidth}
\centering
\input fig_trunc2.tex
\caption{xiaolongbao}
\end{subfigure}
\end{figure}
\end{document}
