\documentclass[../main.tex]{subfiles}
\begin{document}
\section{模型建立和求解}
\subsection{问题1}
\subsubsection{模型准备}
\paragraph{太阳高度角和太阳方位角}
记 \(\alpha _{s}\) 为太阳高度角,\(\gamma _{s}\)为太阳方位角。已知:
\begin{equation}
\sin \alpha _{s} = \cos \delta \cos \varphi \cos \omega + \sin \delta \sin \varphi
\end{equation}
\begin{equation}
\cos \gamma _{s} = \frac{\sin \delta -\sin \alpha_{s} \sin \varphi}{\cos \alpha _{s} \cos \varphi}
\end{equation}
记 \( \varphi\) 为纬度,\(\omega\) 为太阳时角,\(ST\) 为当地时间, \(\delta\) 为太阳赤纬角,有
\begin{equation}
\omega = \frac{\pi}{12} ( ST - 12)
\end{equation}
\begin{equation}
\sin \delta = \sin \frac{2 \pi D}{365} \sin \big(\frac{2\pi}{360} \cdot 23.45\big)
\end{equation}
设太阳中心发出的光线的方向向量为 \(\vec\lambda _{i} = (x,  y , z)\)。若将\(\vec \lambda_{i}\)单位化,容易知道
\begin{equation}
\vec \lambda _{i} = ({-} \cos \alpha _{s} \sin \gamma_{s} ,{-} \cos \alpha_{s} \cos \gamma_{s}, {-} \sin \alpha_{s})
\end{equation}
\paragraph{相关公式}
法向直接辐射照度 DNI 按照下面公式近似计算:
\[
\begin{aligned}
\mathrm{DNI} = G_{0} \bigg[ a + b \cdot \exp\Big\{{-}\frac{c}{\sin \alpha_{s}}\Big\}\bigg]\\
a = 0.4237 + 0.00821 (6 - H) ^{2} \\
b = 0.5055 + 0.00595(6.5 - H) ^{2} \\
c = 0.2711 + 0.01858 (2.5 - H) ^{2}
\end{aligned}
\]
其中 \(G_{0}\) 是太阳常数。镜场的输出热功率为 \(E_{\mathrm{field}}\) 为
\begin{equation}
E_{\mathrm{field}} = \mathrm{DNI} \cdot \sum _{i} ^{N} A_{i} \eta _{i}
\end{equation}
定日镜光学效率\(\eta\)为
\begin{equation}
\eta = \eta _{\mathrm{s b}} \eta _{\cos} \eta _{\mathrm{at}} \eta _{\mathrm{trunc}} \eta _{\mathrm{ref}}
\end{equation}
其中 \(\eta _{\mathrm{s b}}\) 为阴影遮挡效率。阴影遮挡损失是指由于阴影或者是遮挡导致的效率损失,\(\eta _{\cos}\) 是余弦效率,大气透射率 \(\eta _{\mathrm{at}}\) 由下面公式确定:
\begin{equation}
\eta _{\mathrm{at}} = 0.99321 - 0.0001176 d _{\mathrm{HR}} + 1.97 \times 10 ^{-8} \times d _{\mathrm{HR}} ^{2}
\end{equation}
其中 \(d _{\mathrm{HR}}\)是镜面中心到集热器中心的距离。截断效率为\(\eta _{\mathrm{trunc}}\) 。因为部分反射光束没有照射到中心塔塔顶的集热器导致效率损失。

设入射光线为 \(\vec S\),反射光线为 \(\vec R\),余弦效率使用下面公式计算:
\begin{equation}
\eta _{\cos} = \cos \big(\arccos (\vec R \cdot \vec S) / 2\big)
\end{equation}
\paragraph{旋转矩阵}
已知地面坐标系 \(x y z\)以圆形区域的圆心为原点,正北方向为 \(y\) 轴正方向,正东方向为 \(x\) 轴正方向,垂直地面向上的方向为 \(z\) 轴正方向。假设已知一个定日镜的镜面中心在地面坐标系的坐标 \(N = (x_0,  y_0 , d)\),其中 \(d\) 是安装高度,现在以定日镜镜面中心为坐标原点,镜面法向为 \(z\) 轴,如图所示建立坐标系,记为 \(x'y'z'\)。

设 \(\vec n\) 为该镜面的法向量,方向向外,\(\theta\) 为 \(\vec n\) 和 竖直方向上的夹角,\(\varphi\) 是 \(\vec n\) 水平方向和正东方向的夹角。可以得到坐标变换的公式:
\begin{equation}
\begin{pmatrix}
x_{1}\\
y_1\\
z_1
\end{pmatrix}
= M
\begin{pmatrix}
x_2\\
y_2\\
z_2
\end{pmatrix} +
\begin{pmatrix}
x_0\\
y_0\\
d
\end{pmatrix}
\end{equation}
其中 \((x_2,y_2,z_2)\) 是镜面坐标系里点的坐标,\((x_1,y_1,z_1)\) 是地面坐标系里点的坐标,\(M\) 是旋转矩阵,其为
\begin{equation}\label{equ:transM}
M =
% \begin{pmatrix}
% \cos \theta & - \sin \theta & 0 \\
% \sin \theta & \cos \theta & 0\\
% 0 & 0 & 1
% \end{pmatrix} \cdot
% \begin{pmatrix}
% 1 & 0 & 0\\
% 0 & \cos \varphi & - \sin \varphi\\
% 0 & \sin \varphi & \cos \varphi
% \end{pmatrix}
% =
\begin{pmatrix}
\cos \theta & - \cos \varphi & \sin \varphi \sin \theta\\
\sin \theta & \cos \varphi \cos \theta & - \sin \varphi \cos \theta\\
0 & \sin \varphi & \cos \varphi
\end{pmatrix}
\end{equation}
特别地,当 \(\theta = 0\) 的时候,旋转矩阵 \(M\) 退化为单位矩阵。
于是使用定日镜 \(A\)上一点在镜面坐标系 \(x'y'z'\) 的坐标 \(N_{A}\) \((x_{A}, y_{A},0)\),可知其在地面坐标系之中的坐标\(N\) \((x, y , z)\):
\begin{equation}
\begin{pmatrix}
x\\
y\\
z
\end{pmatrix}
=
\begin{pmatrix}
x_{A} \cos \varphi - y_{A} \cos \varphi \sin \theta + x_0\\
x_{A} \sin \theta + y_{A} \cos \varphi \cos \theta + y_{0} \\
y_{A} \sin \varphi + d
\end{pmatrix}
\end{equation}


%%
\paragraph{阴影遮挡效率的计算}
阴影遮挡效率 \(\eta _{\mathrm{s b}}\) 在进行定日镜光学效率的计算之中计算比较复杂的部分。为了计算阴影遮挡效率,需要计算定日镜上某点是否有阴影遮挡损失,而阴影遮挡损失分为以下三种情况:
% TODO cite require
\begin{enumerate}
\item 入射光线被塔挡住
\item 入射光线被其他定日镜挡住(阴影)
\item 反射光线被其他定日镜挡住(挡光)
\end{enumerate}
对于定日镜 A,建立相对于定日镜A的坐标系,以镜面中心为原点水平方向为 \(x\) 轴,平行于镜面、垂直于 \(x\) 轴方向为 \(y\) 轴,以垂直于镜面向上为 \(z\) 轴。
已知定日镜 A 上的一点 \(N_{A} = (x_{A}, y_{A})\)和入射光线 \(\vec {\lambda _{i}}\)。

Step1. 判断入射光线是否被中心塔遮挡,需将定日镜 A 坐标系之中的点 \(N_{A}\) 转化为相对于地面的坐标 \(N\),随后根据 \(\vec {\lambda _{i}}\),计算入射光线的反向延长线与中心塔是否有交点,
为此,可以直接联立直线方程和中心塔圆柱方程,查看是否有解。

由于直接联立方程计算量大,为了简化计算,本文根据入射光线和中心塔塔顶所在水平面的交点 \(N_{h} = (x _{h} , y_{h}, h)\)判断是否有遮挡。

下面给出几何模型。过点 \(N\) 作平面和中心塔圆柱相切,设相切作成的直线和中心塔塔顶所在平面的交点为 \(R\)。设中心塔在塔顶所在平面的圆的圆心为 \(O\),\(A\) 为\(N\)在中心塔塔顶所在平面内的投影,\(\vec \lambda _{i} '\)为 \(\vec \lambda _{i}\) 在平面内的投影。如下图所示
\input fig_ta.tex
\begin{enumerate}
\item 若是入射光线水平方向的夹角过大,也即
\begin{equation}
\cos \langle \overrightarrow {OA}, \vec {\lambda _{i} '} \rangle \le \frac{\vert AR \vert }{\vert OA \vert }
\end{equation}
则不会被遮挡;
\item 若是 \(N_{h}\) 更加靠近定日镜一侧,也即
\begin{equation}
\vert {\it A N}_{h}\vert \le \vert {\it AR\,} \vert
\end{equation}并且 \(N_{h}\) 不在圆内,则不会被遮挡;
\item 否则入射光线会被遮挡。
\end{enumerate}

Step2. 判断入射光线是否被相邻定日镜挡住

为了简化计算,假设发生阴影的两面定日镜相互平行,采用投影法。给定两个定日镜 \(A, B\),将 \(B\) 镜投影到镜面 \(A\) 上,为此,先计算 \(B\) 的中心投影到 \(A\) 上的关于 \(A\) 镜面系的坐标。
% TODO

Step3. 判断反射光线是否被相邻定日镜挡住
已知反射光线 \(\vec \lambda _{o} = (x_0,y_0,z_0)\)和镜面 \(A\) 上反射点在地面坐标系的坐标 \(N\)\((x, y, z)\)。联立反射光线的直线方程和B镜面的平面方程,可以解得光线和 B 的交点在地面坐标系下的坐标 \(N'\)。

随后运用 equ~(\ref{equ:transM}) 将 \(N'\) 坐标转换为 \(B\) 镜面系下的坐标 \(N_{B}\):
\[
N_{B} = M (N' - O)
\]
其中 \(O = (x_{B} , y_{B} , d)\),也即B镜面的中心。

%%
\paragraph{光锥}
因为太阳光并不是平行光,在定日镜上的每个点发出的反射光实际是一个光锥,其半角展宽为 \(4.65 \, \mathrm{m}\mathrm{rad}\),其值可以通过计算太阳半径除以地日距离求得,%
% TODO 半角展宽的推导
在太阳发散角内均匀追迹若干光线作为入射光。又因为定日镜是平面镜,于是入射光和反射光完全对称。

以太阳中心发出的光线的反射光 \(\vec\lambda _{o} \) 为 \(z\) 轴,以镜面上的反射点 \(N\) 为远点,建立关于 \(\vec \lambda _{o}\) 的光锥坐标系。
% TODO 光锥 fig require

对于光锥内的一束光线 \(\vec S _{0}\),记 \(\tau\) 为光线和 \(\vec\lambda _{o}\) 所成的角,其取值范围为 \([0, 4.65\, \mathrm{m}\mathrm{rad}]\),\(\sigma\) 为光线和 \(X\) 轴所成的角,其取值范围为 \([0 , 2\pi ]\),于是光线的坐标可以表示为:
\[
\vec S _{0} = (\sin \sigma \cos \tau, \sin \sigma \sin \tau, \cos \sigma)
\]
通过变换矩阵,可以将光线在光锥坐标系下的坐标转换为地面坐标系下的坐标 \(\vec S\)。

为此,先确定光锥坐标系的轴线的单位向量在地面坐标系下的坐标表示,记为 \(\hat x, \hat y, \hat z\),\(\hat z\) 为 \(\vec \lambda _{o}\)。可以联立
\[
\begin{cases}
\hat x \cdot \hat z = 0\\
\hat y = \hat z \times \hat x
\end{cases}
\]
若是设 \(\hat x = (m , n , 0)\),则有
\[
\begin{cases}
\displaystyle\hat x = \left(\frac{\vert y_{0} \vert}{\sqrt{x_{0}^{2}+ y _{0} ^{2}}}, {-}\frac{x_{0}\vert y_{0} \vert}{y_{0}\sqrt{x_{0}^{2}+y_{0}^{2}}}, 0\right)\\
\displaystyle \hat y = ({-} n z_{0}, - m z_{0} , my_{0} - n x_{0})
\end{cases}
\]
特别地,若是 \(y_{0} = 0\) 则 \(m = 0 \), \(n=1\),也即 \(\hat x  = (0,1,0)\)
若将 \(\vec S_{0}\) 记为 \((a,b ,c)\),可得
\begin{equation}
\vec S = (am - b nz _{0}+ c x_{0}, an - bmz_{0} + cy_{0}, bmy_{0} - bnx _{0} + cz_{0})
\end{equation}

%%
\paragraph{截断效率的计算}
% TODO 补充 require
计算截断效率,需要确定光锥之中的某光线是否有截断损失,若是其没有打在集热器上,则该光线发生了截断损失。为了判断截断损失,采取的思路和上文确定中心塔造成的阴影损失的思路类似。

已知反射点 \(N\) 和 反射光线 \(\vec S\),确定光线 1) 和中心塔塔顶所在平面的交点 \(N'\);2) 和集热器底部所在的平面的交点 \(N''\)。剩余符号沿用上文,如下图所示:

\begin{figure}[H]
\centering
\begin{subfigure}[b]{0.4\textwidth}
\centering
\input fig_trunc.tex
\caption{竖直面示意图}
\end{subfigure}
\begin{subfigure}[b]{0.4\textwidth}
\centering
\input fig_trunc2.tex
\caption{水平面示意图}
\end{subfigure}
\caption{判断截断损失的示意图}
\end{figure}

分三个步骤进行截断损失的判断:
\begin{enumerate}
\item 若是 \( \displaystyle\cos \langle \overrightarrow{OA}, \vec S' \rangle\le \frac{\vert AR \vert}{\vert OA \vert}\),则发生截断损失
\item 若是 \(\vert AN'\vert \le \vert AR \vert\)且 \(N'\) 不在圆内,则发生截断损失
\item 若是 \(N''\) 在圆 \(O\) 内,或是 \(\vert AN'' \vert \ge \vert AR \vert\)
\end{enumerate}


\subsubsection{模型建立}
%%
计算法向量
针对问题中的任意一个日期\(D\)和时间点\(ST\),由公式(1)(2)(3)(4)(5)
% equ require
可以确定在此时刻太阳中心发出的光线,即所有定日镜的入射光线的方向向量\(\vec \lambda_i=(x_i,y_i,z_i)\)。

接下来对所有定日镜进行遍历,根据定日镜中心
\(M(x_0,y_0,d)\)以及吸收塔中心的坐标\((0,0,h)\)可以确定入射光线经定日镜反射后的反射光线的方向向量
为 \(\vec \lambda_o (−x_0,−y_0, d −h)\),根据向量之间的共面关系以及
法向量是入射光线、反射光线的角平分线,通过公式
\begin{equation}
\vec n = \big(\frac{\vec \lambda_{i}}{\vert \lambda_{i} \vert} + \frac{\vec \lambda_{o}}{\vert \lambda_{o} \vert}\big)
\end{equation}
即可确定每个定日镜镜面的法向量 \(\vec n (x_{n}, y_{n}, z_{n})\)

%%
蒙特卡罗模拟生成随机点

定日镜场光学效率是所有单个定日镜的光学效率的均值,所以本文先对所有单个定日镜的光学效率进行遍历求解。

由于是否受到遮挡的情况比较复杂,遮挡部分的阴影面积较难求解,本文在此处使用蒙特卡罗模拟方法,在定日镜内以定日镜中心为原点,水平方向为\(x\)轴的二维直角坐标系中随机生成一个反射点\(N(x_A,y_A)\)

%%
计算阴影遮挡效率\(\eta_{\mathrm{s b}}\)\par
入射方向\(\vec \lambda _{i}\) 以及反射点的法向量\(\vec n\)已知,根据前后定日镜的位置,以及吸收塔的位置和高度,从而可以确定反射点是否被其他物件遮挡住,进一步确定该点是否是阴影。根据蒙特卡罗原理,通过大量的随机点的试验,阴影遮挡效率就是未被挡住的点的数量与试验点总数的比值。

对于\(\eta _{\mathrm{s b}}\)的具体计算过程,本文对阴影遮挡损失的三种情况进行如下分类:
% TODO cite require
\begin{enumerate}
\item 入射光线被塔挡住
\item 入射光线被其他定日镜挡住(阴影)
\item 反射光线被其他定日镜挡住(挡光)
\end{enumerate}
对于定日镜 A,建立相对于定日镜A的坐标系,以镜面中心为原点水平方向为 \(x\) 轴,平行于镜面、垂直于 \(x\) 轴方向为 \(y\) 轴,以垂直于镜面向上为 \(z\) 轴。
已知定日镜 A 上的一点 \(N_{A} = (x_{A}, y_{A})\)和入射光线 \(\vec {\lambda _{i}}\)。

Step1. 判断入射光线是否被中心塔遮挡,需将定日镜 A 坐标系之中的点 \(N_{A}\) 转化为相对于地面的坐标 \(N\),随后根据 \(\vec {\lambda _{i}}\),计算入射光线的反向延长线与中心塔是否有交点,
为此,可以直接联立直线方程和中心塔圆柱方程,查看是否有解。

由于直接联立方程计算量大,为了简化计算,本文根据入射光线和中心塔塔顶所在水平面的交点 \(N_{h} = (x _{h} , y_{h}, h)\)判断是否有遮挡。

下面给出几何模型。过点 \(N\) 作平面和中心塔圆柱相切,设相切作成的直线和中心塔塔顶所在平面的交点为 \(R\)。设中心塔在塔顶所在平面的圆的圆心为 \(O\),\(A\) 为\(N\)在中心塔塔顶所在平面内的投影,\(\vec \lambda _{i} '\)为 \(\vec \lambda _{i}\) 在平面内的投影。如下图所示
\input fig_ta.tex
\begin{enumerate}
\item 若是入射光线水平方向的夹角过大,也即
\begin{equation}
\cos \langle \overrightarrow {OA}, \vec {\lambda _{i} '} \rangle \le \frac{\vert AR \vert }{\vert OA \vert }
\end{equation}
则不会被遮挡;
\item 若是 \(N_{h}\) 更加靠近定日镜一侧,也即
\begin{equation}
\vert {\it A N}_{h}\vert \le \vert {\it AR\,} \vert
\end{equation}并且 \(N_{h}\) 不在圆内,则不会被遮挡;
\item 否则入射光线会被遮挡。
\end{enumerate}

Step2. 判断入射光线是否被相邻定日镜挡住

由于定日镜场的地理位置位于北纬\(39.4^{\circ}\),太阳光线在一年中的沿南北方向的分量一定是由南向北的,因此在地面坐标系\(xyz\)下定日镜A只能被南面的定日镜挡住光线。本文通过分析定日镜场每一圈之间的间隙长度\(d_{\mathrm{belt}}\),以及定日镜场中各点的分布(graph1.2)和相邻定日镜之间的排布间隙要比镜面宽度\(a\)至少大\(5 \, \mathrm{m}\)的要求
% TODO graph require

确定了一个定日镜A的邻域:\((x_B - x_A)^2 + (y_B - y_A)^2 \le \epsilon^2\),其中\(\varepsilon^2=d_{\mathrm{belt}}^2+(a+5)^2\)。

通过上述邻域筛选得到可能挡住定日镜A入射光线的定日镜群,接下来对群中的每一个定日镜进行如下分析:

给定两个定日镜 \(A, B\),将 \(B\) 镜投影到镜面 \(A\) 上,为此,先计算 \(B\) 的中心投影到 \(A\) 上的关于 \(A\) 镜面系的坐标。

设\(A,B\) 的中心坐标分别为 \((x_{A}, y_{A}, d)\),\((x_{B}, y_{B}, d)\),设入射光线的方向向量 \(\vec \lambda _{i} = (x_{i}, y_{i}, z_{i})\),由于相邻定日镜之间的距离较近,且镜面朝向之间的差距不搭,本文假设发生阴影遮挡的两定日镜的镜面相互平行,于是设它们的法向量均为 \(\vec n\)。

可知从定日镜B镜面中心射出的光线的直线方程为
\[
\begin{cases}
x = x _{B} + m x_{i}\\
y = y _{B} + my_{i} \\
z = z_{B} + m z_{i}
\end{cases}
\]
定日镜 A 镜面的平面方程为
\[
(x - x_{A} , y- y _{A} , z - d) \cdot \vec n = 0
\]
联立上面直线方程和平面方程即可得到定日镜B的中心在定日镜A镜面上的投影点 \(B' (x_{B}' , y_{B} ' , z _{B} ')\)。由于得到的坐标是地面坐标系下的坐标,还需使用 equ~(\ref{transM}) 将其转化为 A 镜面系下的坐标:
\[
\begin{pmatrix}
x_{B} ''\\
y_{B} ''\\
z _{B} ''
\end{pmatrix}
=
\begin{pmatrix}
\cos \theta & \sin \theta & 0\\
-\cos \varphi \sin \theta & \cos \varphi & \sin \varphi \\
\sin \varphi \sin \theta & - \sin \varphi \cos \theta & \cos \varphi
\end{pmatrix}
\cdot
\begin{pmatrix}
x_{B} ' - x_{A}\\
y_{B}' - y_{A} \\
z_{B} ' - d
\end{pmatrix}
\]
% TODO
Step3. 判断反射光线是否被相邻定日镜挡住

已知反射光线 \(\vec \lambda _{o} = (x_0,y_0,z_0)\)和镜面 A 上反射点在地面坐标系的坐标 \(N'\)\((x_1, y_1, z_1)\),以及镜面 B 的镜面法向量\(\vec n = (x_{n} , y_{n} ,z_{n})\)。类似的,联立反射光线的直线方程和B镜面的平面方程,
反射光线的直线方程为
\[
\begin{cases}
x = x_1 + m x_0\\
y = y_1 + m y_0\\
z = z_1 + m z_0
\end{cases}
\]
定日镜 B镜面的平面方程为
\[
(x - x_{B} , y - y_{B} , z-z_{B}) \cdot \vec n = 0
\]
可以解得光线和 B 的交点在地面坐标系下的坐标 \(N''\)为
\[
(x_2 , y_2, z_2) = (x_{A} + k x_0, y_{A} + ky_{0} , z_{A} + kz_{0})
\]
其中\(k = \displaystyle \frac{(x_{B} - x_{A}) x_{n} + (y_{B} - y_{A}) y_{n} + (d- z_{A}) z_{n}}{x_0 x_{n} + y_{0} y_{n} + z_{0} z _{n}}\)。

随后运用 equ~(\ref{equ:transM}) 将 \(N''\) 坐标转换为 \(B\) 镜面系下的坐标 \(N_{B}\):
\[
\begin{pmatrix}
x_{B} '\\
y_{B} '\\
z _{B} '
\end{pmatrix}
=
\begin{pmatrix}
\cos \theta & \sin \theta & 0\\
-\cos \varphi \sin \theta & \cos \varphi & \sin \varphi \\
\sin \varphi \sin \theta & - \sin \varphi \cos \theta & \cos \varphi
\end{pmatrix}
\cdot
\begin{pmatrix}
x_{A} + kx_0 - x_{B}\\
y_{A} + ky_{0} - y_{B}\\
z_{A} + kz_{n} - d
\end{pmatrix}
\]
由此容易判断出 \(N_{B}\) 是否在镜面B之中,也就能够判断出反射光线是否会被B镜面遮挡。

至此可以判断对于某个随机点试验,其是否被遮挡,\(N_{\text{被挡次数}}\)是否要增加。

若最终得出的结果是随机反射点属于阴影部分,因为截断效率\(\eta _{\mathrm{trunc}}\)在问题中的定义如下:
% pass 3 require
\begin{equation}
\eta _{\mathrm{s b}} =
\end{equation}
那么该点的截断效率就失去了意义,则跳过下一步骤。
如果随机反射点没有被挡住,则继续计算其截断效率。

%%
为了计算该反射点的截断效率\(\eta _{\mathrm{dot}{-}\mathrm{trunc}}\),此处需要考虑到在反射点反射出的光线是一束具有锥形角的一束锥形光线:
% guangzhui require

%%
截断效率的计算: 此处继续使用蒙特卡罗模拟,在上述的以反射点为顶点,主反射光线为轴的光锥中随机生成具有偏移量为\(\tau\),\(\sigma\)的某一角度的试验反射光线(向量 \(\vec S\)),通过计算打在吸收塔集热器上的反射光线数与总试验次数的比例,最终获得该反射点对应的截断效率\(\eta _{\mathrm{dot}{-}\mathrm{trunc}}\)。

为了。。。。类似。
% TODO
通过对于每个定日镜中所有试验点是否被遮挡和其截断效率(部分试验点位于阴影部分,没有这个系数),对于这个定日镜,
其最终的阴影遮挡效率\(\eta _{\mathrm{s b}}\)计算就可以由以下公式得出:
% pass 2 require
\begin{equation}
\eta _{\mathrm{trunc}} = \sum \frac{\eta _{\mathrm{s b}}}{分子}
\end{equation}

%%
其余各光学效率的计算

通过公式(8)(9)计算得到\(\eta _{\mathrm{at}}\)以及\(\eta _{\cos}\)
通过公式(7)得到该定日镜的光学效率\(\eta_{i}\)
% equ require

定日镜场 平均光学效率等平均指标 就是所有定日镜光学效率的均值
最终通过公式(6)即可求得镜场输出热功率
% equ require

最终求得的结果展示如下:

% Table1
% Table2
% Figure_A1picture
\end{document}
